% Preamble
% ---
\documentclass{article}


% Packages
% ---
%\usepackage{amsmath} % Advanced math typesetting
\usepackage[utf8]{inputenc} % Unicode support (Umlauts etc.)
\usepackage[ngerman]{babel} % Change hyphenation rules
\usepackage{hyperref} % Add a link to your document
\usepackage{graphicx} % Add pictures to your document
\usepackage{listings} % Source code formatting and highlighting
\usepackage{caption}
\usepackage{subcaption}

% custom commands
\newcommand{\quotes}[1]{``#1''}

\graphicspath{ {./img/} }

\begin{document}

    \author{Federico Rachelli}
    \title{\vspace{-2cm}DataVirus.it}
    \maketitle
    
    \section{Funzionamento dell'applicazione}

    L'app \textbf{\href{https://datavirus.it}{DataVirus.it}} é ispirata dal sito web omonimo.
    Fornisce all'utente una visualizzazione grafica dei dati del Dipartimento di Protezione Civile sull'andamento dell'epidemia di COVID-19.
    \\
    I dati sono accedibili dal \href{https://github.com/pcm-dpc/COVID-19}{repository GitHub ufficiale} della Protezione Civile. 
    Tali dati sono aggiornati quotidianamente (a partire dalle ore 18:00) fino alla fine dello stato di emergenza dichiarato dal Consiglio dei Ministri in data 31 Gennaio 2020.
    
    \subsection{Visualizzazione dati numerici}
    L'applicazione all'avvio visualizza una schermata di caricamento dei dati dal Dipartimento di Protezione Civile. 
    Tali dati, una volta ottenuti, sono elaborati e vengono mostrati all'utente gli andamenti a livello nazionale dell'epidemia.
    
    \begin{figure}[h]
        \centering
        \begin{subfigure}{.5\textwidth}
          \centering
          \includegraphics[width=.7\linewidth]{loading_dialog.jpg}
          \caption{Caricamento dei dati}
          \label{fig1:sub1}
        \end{subfigure}%
        \begin{subfigure}{.5\textwidth}
          \centering
          \includegraphics[width=.7\linewidth]{main_activity.jpg}
          \caption{Andamento nazionale}
          \label{fig1:sub2}
        \end{subfigure}
    \end{figure}
    
    Come si puó notare nella figura \ref{fig1:sub2}, questa é la schermata principale dell'applicazione. 
    Viene mostrato in alto la denominazione territoriale dei dati (nella scermata principale verrá mostrato l'andamento nazionale). Sotto i pulsanti, viene visualizzata la data dell'ultimo aggiornamento disponibile dal Dipartimento di Protezione Civile.
    \\
    Il pulsante \quotes{Aggiorna} ricarica i dati dal repository.
    \\
    Vengono poi visualizzate le tile contenenti il dato odierno e la relativa variazione (delta) rispetto alla giornata precedente.
    \\
    
    \subsection{Cambio zona geografica}
    La denominazione geografica puó essere cambiata premendo sul pulsante \quotes{Nuova localitá} dalla schermata principale (vedi \ref{fig1:sub2}).
    \\
    Appare il picker della zona geografica:

    \begin{figure}[h]
        \centering
        \includegraphics[width=.5\linewidth]{DPC_geo_picker.png}
        \caption{Picker zona geografica}
        \label{fig2}
    \end{figure}

    Grazie al picker si puó scegliere la zona geografica di interesse. Puó essere una provincia, una regione oppure si puó selezionare l'andamento nazionale.
    Per comoditá si puó ulteriormente cercare la zona d'interesse premendo sulla \emph{lente d'ingrandimento}.
    Come risultato verranno visualizzati i dati relativi alla zona selezionata:

    \begin{figure}[h]
        \centering
        \includegraphics[width=.5\linewidth]{lombardia.png}
        \caption{Esempio visualizzazione regione Lombardia}
        \label{fig3}
    \end{figure}

    Premendo il pulsante back \emph{fisico} si scorreranno all'indietro tutte le selezioni geografiche finora cercate dall'utente.

    \subsection{Preferiti}
    Ogni tile ha in alto a destra una \emph{stellina}: quando premuta, la tile relativa viene salvata tra i preferiti.
    Per recuperare la lista delle tile salvate é sufficiente premere sul floating button in basso a destra (visibile nella schermata principale dell'app, \ref{fig1:sub2}).

    \begin{figure}[h]
        \centering
        \includegraphics[width=.5\linewidth]{preferences.png}
        \caption{Lista dei preferiti}
        \label{fig4}
    \end{figure}

    Come si puó notare dalla figura \ref{fig4}, le tile salvate sono visualizzate con la loro denominazione geografica a seguito.
    Premendo su una \emph{stellina} precedentemente marcata verrá rimossa tale tile dalla lista dei preferiti.

    \subsection{Grafici}
    Ogni qualvolta si premerá su una tile, verrá mostrato l'andamento nel tempo del dato selezionato in un grafico.
    Si presenterá la schermata come quella in figura \ref{fig5:sub1}.
    \\
    Il dato che ora viene mostrato puó essere confrontato con altri dati. Per ottenere questo risultato sará sufficiente premere sul pulsante \quotes{Aggiungi} e selezionare una nuova tile.
    \\
    Ad ogni dato aggiunto viene assegnato un colore random per la sua rappresentazione e apparirá nella lista dei campi attualmente disegnati. 
    Per ciascuno di questi si potrá decidere se nasconderlo o mostrarlo (con la checkbox sulla sinistra) oppure cancellare dalla lista premendo sull'icona del \emph{cestino}.
    \\
    Il grafico é ingrandibile a piacimento, semplicemente utilizzando il \emph{pinch-to-zoom}.

    \begin{figure}[h]
      \centering
      \begin{subfigure}{.5\textwidth}
        \centering
        \includegraphics[width=.7\linewidth]{milano.png}
        \caption{Esempio grafico Milano}
        \label{fig5:sub1}
      \end{subfigure}%
      \begin{subfigure}{.5\textwidth}
        \centering
        \includegraphics[width=.7\linewidth]{chart_ICU.jpg}
        \caption{Confronto tra terapie intensive}
        \label{fig5:sub2}
      \end{subfigure}
    \end{figure}
    Alla pressione del pulsante \quotes{Pulisci} si chiuderá la schermata del grafico e il grafico attualmente disegnato verrá scartato, insieme a tutti i dati precedentemente immessi.
    Il medesimo risultato si puó raggiungere alla pressione del pulsante back \emph{fisico}.

    \subsection{Notifiche periodiche}
    Nell'activity principale, a sinistra della data di ultimo aggiornamento, é presente una checkbox con a fianco una campanellina (vedere fig. \ref{fig1:sub2}).
    \\
    Quando marcata, a partire dalle ore 18:00 di ogni giorno, ogni 10 minuti verrá performata una ricerca di nuovi dati aggiornati (orario nel quale il DPC li rende pubblici).
    Nel caso in cui i dati scaricati abbiano data odierna, l'utente verrá avvisato attraverso una notifica. Alla pressione di tale notifica si aprirá l'app.

\end{document}